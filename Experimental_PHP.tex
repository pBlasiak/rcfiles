\documentclass[review]{elsarticle}

\usepackage{lineno,hyperref}
\modulolinenumbers[5]

%\usepackage[english]{babel}
%\usepackage[utf8x]{inputenc}
%\usepackage[T1]{fontenc}
%
\usepackage{threeparttable}
\usepackage{graphicx}
\usepackage{subfig}
\usepackage{setspace}
%% potrzebne do tabeli jak w przykładzie
\usepackage{tabularx}
\newcolumntype{C}{>{\centering\arraybackslash}X}
%
\usepackage{epstopdf}
\usepackage{amssymb}
\usepackage{amsmath}
\usepackage{tabularx}
\usepackage{multirow}
\usepackage{float}
\restylefloat{table}
%%\usepackage{multirow}
\usepackage{booktabs}
%\usepackage{authblk}
\usepackage[bottom]{footmisc}
%\usepackage[figurename=Fig.]{caption}
\usepackage[section]{placeins}
\usepackage{color, soul} % highlighting
\newcolumntype{Y}{>{\centering\arraybackslash}X} %necessary for tabularx to make equal width columns
\usepackage[usenames,dvipsnames]{xcolor} 
\usepackage[draft]{todonotes}   % notes showed; should be after xcolor

%\newcommand{\hlc}[2][yellow]{ {\sethlcolor{#1} \hl{#2}} }



\journal{Applied Energy}



%%%%%%%%%%%%%%%%%%%%%%%
%% Elsevier bibliography styles
%%%%%%%%%%%%%%%%%%%%%%%
%% To change the style, put a % in front of the second line of the current style and
%% remove the % from the second line of the style you would like to use.
%%%%%%%%%%%%%%%%%%%%%%%

%% Numbered
%\bibliographystyle{model1-num-names}

%% Numbered without titles
%\bibliographystyle{model1a-num-names}

%% Harvard
%\bibliographystyle{model2-names.bst}\biboptions{authoryear}

%% Vancouver numbered
%\usepackage{numcompress}\bibliographystyle{model3-num-names}

%% Vancouver name/year
%\usepackage{numcompress}\bibliographystyle{model4-names}\biboptions{authoryear}

%% APA style
%\bibliographystyle{model5-names}\biboptions{authoryear}

%% AMA style
%\usepackage{numcompress}\bibliographystyle{model6-num-names}

%% `Elsevier LaTeX' style
\bibliographystyle{elsarticle-num}
%%%%%%%%%%%%%%%%%%%%%%%

\begin{document}

\begin{frontmatter}

\title{...}
 
 \author[pwr]{first}
 
 \author[pwr]{second}
   
 \author[pwr]{S\l awomir Pietrowicz\corref{cor1}}
 \ead{slawomir.pietrowicz@pwr.edu.pl}
 
 \cortext[cor1]{Corresponding author Tel.: +48 71 320 36 17}
 
 \address[pwr]{Department of Thermodynamics, Theory of Machines and Thermal Systems, Faculty of Mechanical and Power Engineering, Wroc\l aw University of Technology, Wroc\l aw Wybrze\.ze Wyspia\'nskiego 27, Poland}


\begin{abstract}

\end{abstract}

\begin{keyword}
Oscillating Heat Pipe \sep  \sep  \sep 
\end{keyword}

\end{frontmatter}

\linenumbers

\section*{Nomenclature}

\begin{tabular}{l c p{10.5cm}}

  $A_c$ & -- & cross-sectional area of the tube, $\mathrm{m^2}$ \\
       
\end{tabular} 
\\ 

\textit{ Greek symbols}

\begin{tabular}{l c p{10.5cm}}

  $\rho$ & -- & density, $\mathrm{kg/m^3}$ \\ 
       
\end{tabular} 
\\ 
%\textit{ Superscripts}

\textit{ Subscripts}

\begin{tabular}{l c p{10.5cm}}
 
  $l$ & -- & liquid \\
  $v$ & -- & vapor \\
  $h$ & -- & heating \\
  $c$ & -- & condensing section \\
  $e$ & -- & evaporation section \\
       
\end{tabular} 
\\

\textit{Abbreviations}

\begin{tabular}{l c p{10.5cm}}

  $PHP$ & -- & Pulsating Heat Pipe \\  
       
\end{tabular} 

\section{Introduction}
\label{sec:intro}

\section{Experimental setup}
\subsection{Preparation}
\subsection{Podzespoły}
\subsection{Dokładność}
\section{Results and discussion}
\label{sec:resdis}
\section{Conclusions}
\section*{Acknowledgements}
The work was financed by Grant No.POiR.04.01.04-00-0037/15. Calculations have been carried out using resources provided by Wroclaw Centre for Networking and Supercomputing (http://wcss.pl).

\section*{References} 

\bibliography{PHP}

\end{document}
